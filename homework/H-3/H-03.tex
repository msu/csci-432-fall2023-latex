\documentclass{article}
\usepackage{../fasy-hw}

%% UPDATE these variables:
\renewcommand{\hwnum}{3}
\title{Advanced Algorithms, Homework \hwnum}
\collab{none}
\author{\todo{Your Name Here}}
\date{due: Friday, 27 October 2023}

\begin{document}

\maketitle

This homework assignment should be
submitted as a single PDF file both to D2L and to Gradescope.  The homework is
due at 23:59 on the due date; however, you can continue to submit until the
instructor begins grading, at which point no more submissions will be accepted.

General homework expectations:
\begin{itemize}
    \item Homework should be typeset using LaTex.
    \item Answers should be in complete sentences and proofread.
    \item You will not plagiarize, nor will you share your written solutions
        with classmates.
    \item List collaborators at the start of each question using the
        \texttt{collab} command. If this is still a `TODO' at the time of
        submission, the problem will not
        be graded.
    \item List any outside resources that you used within each problem (not at
        the end of the problem set). If not clear from context, explain how the
        external resource was used.  Outside resources include the use of
        ChatGPT, googling definitions, and looking up answers after you solve
        the problem in order to `check that you got it right'.  {\bf If no outside resources are
        used, the last line of your answer should be ``No outside resources were
        used when working on this problem.''}  Note: in general, the assumption
        is that you do not need any outside resources to answer the problems.
        If you find yourself needing extra resources, it is highly advised to
        ask for them rather than to find them on your own.
    \item Put your answers where the \texttt{todo} command currently is (and
        remove the \texttt{todo}, but not the word \texttt{Answer}).
    \item If you are asked to come up with an algorithm, you are
        expected to give an algorithm that beats the brute force (and, if possible, of
        optimal time complexity). With your algorithm, please provide:
        \begin{itemize}
            \item \emph{What}: A prose explanation of the problem and the algorithm,
                including a description of the input/output.
            \item \emph{How}: Describe how the algorithm works, including giving
                psuedocode for it.  Be sure to reference the pseudocode
                from within the prose explanation.
            \item \emph{How Fast}: Runtime, along with justification.  (Or, at
                the very least, a proof of termination using a decrementing function).
            \item \emph{Why}: Briefly explain why the algorithm works.  Be sure
                to include a statement of the loop invariant for each loop, or
                recursion invariant for each recursive function.
        \end{itemize}
    \item Remove the set of instructions on this page (everything from Line 15
        through Line 59 in the LaTex file.
\end{itemize}

%%%%%%%%%%%%%%%%%%%%%%%%%%%%%%%%%%%%%%%%%%%%%%%%%%%%%%%%%%%%%%%%%%%%%%%%%%%%%%
\collab{\todo{}}
\nextprob{Recursive Definitions}

Chapter 1, Problem 4.  Note that this is just asking for the recursive
definition, so you do not need to provide the What/How/How Fast/Why that you
would when presenting a full algorithm.

\paragraph{Answer}
\todo{}


\end{document}
