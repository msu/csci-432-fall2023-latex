\documentclass{article}
\usepackage{../fasy-hw}

%% UPDATE these variables:
\renewcommand{\hwnum}{1}
\title{Advanced Algorithms, Homework \hwnum}
\collab{none}
\author{\todo{Your Name Here}}
\date{due: Friday, 15 September 2022}

\begin{document}

\maketitle

This homework assignment should be
submitted as a single PDF file both to D2L and to Gradescope.

General homework expectations:
\begin{itemize}
    \item Homework should be typeset using LaTex.
    \item Answers should be in complete sentences and proofread.
    \item You will not plagiarize, nor will you share your written solutions
        with classmates.
    \item List collaborators at the start of each question using the
        \texttt{collab} command.
    \item List any outside resources that you used within each problem (not at
        the end of the problem set). If not clear from context, explain how the
        external resource was used.
    \item Put your answers where the \texttt{todo} command currently is (and
        remove the \texttt{todo}, but not the word \texttt{Answer}).
\end{itemize}

%%%%%%%%%%%%%%%%%%%%%%%%%%%%%%%%%%%%%%%%%%%%%%%%%%%%%%%%%%%%%%%%%%%%%%%%%%%%%%
\collab{\todo{list your collaborators here}}
\nextprob{The Course Website and Syllabus}

Please take a look at the \href{https://msu.github.io/csci-432-fall2023/}{course
website}.

Answer the following questions:
\begin{enumerate}

    \item This website is made using GitHub Pages.  Find one commit message and
        provide the commit hash, the author of the commit, and the files edited.

         \paragraph{Answer} \todo{}

    \item What are Dr.~Fasy's office hours?

         \paragraph{Answer} \todo{}

    \item Where can you find links to the lecture notes?

         \paragraph{Answer} \todo{}

    \item The section on the use of ChatGPT is currently blank.  What is your
        suggestion for appropriate use of ChatGPT in this class?

         \paragraph{Answer} \todo{}

    \item In the last HW, you identified your weaknesses from the prerequisite
        material.  What have you been doing to dust off those cobwebs?

         \paragraph{Answer} \todo{}

     \item (Bonus) See an error?  Issue a PR!

         \paragraph{Answer} \todo{state whether you issued a PR or not}
\end{enumerate}

%%%%%%%%%%%%%%%%%%%%%%%%%%%%%%%%%%%%%%%%%%%%%%%%%%%%%%%%%%%%%%%%%%%%%%%%%%%%%%
\collab{\todo{}}
\nextprob{Induction}

There are a couple series that we see consistently in many algorithms.  One of
them is the following:
$$
    \sum_{i=1}^n i
$$

State the solution in both \emph{closed form} and the \emph{asymptotic form}.
Then, use induction to prove that the asymptotic form is correct
(Note: when giving asymptotic form, the preference is always to
provide the bound using $\Theta$ notation, if possible).

\paragraph{Answer}
\todo{}

%%%%%%%%%%%%%%%%%%%%%%%%%%%%%%%%%%%%%%%%%%%%%%%%%%%%%%%%%%%%%%%%%%%%%%%%%%%%%%
\collab{\todo{}}
\nextprob{Big-O}
Use the definition of big-O notation to prove that $f(x)=n^2 + 3n -2$ is
$O(n^2)$.

\paragraph{Answer}

\todo{}

%%%%%%%%%%%%%%%%%%%%%%%%%%%%%%%%%%%%%%%%%%%%%%%%%%%%%%%%%%%%%%%%%%%%%%%%%%%%%%
\collab{\todo{}}
\nextprob{If/Then Statements}
Consider the following statement: \emph{If $a$ and $b$ are both even numbers, then $ab$ is
an even number}.
\begin{enumerate}
    \item What is the contrapositive of this statement?

        \paragraph{Answer}
        \todo{}

    \item What is the converse of this statement?

        \paragraph{Answer}
        \todo{}

    \item What is the definition of an odd number?

        \paragraph{Answer}
        \todo{}

    \item What is the definition of an even number?

        \paragraph{Answer}
        \todo{}

    \item Prove this statement.

        \paragraph{Answer}
        \todo{}

\end{enumerate}


%%%%%%%%%%%%%%%%%%%%%%%%%%%%%%%%%%%%%%%%%%%%%%%%%%%%%%%%%%%%%%%%%%%%%%%%%%%%%%
\collab{\todo{}}
\nextprob{Sorting}
Consider your favorite sorting algorithm.
\begin{enumerate}
    \item \emph{What} is the problem that this algorithm solves?

        \paragraph{Answer}
        \todo{}

    \item \emph{How} does it work? (Hint: give pseudocode!)

        \paragraph{Answer}
        \todo{}

    \item \emph{How fast} does it work?  Give the asymptotic running time.
        Note: typically, you will give this as the worst-case running time.
        However, if you chose quicksort or another randomized algorithm, please
        give both the worst-case running time and the expected running time.  No
        justification of the running time is needed here.

        \paragraph{Answer}
        \todo{}

    \item \emph{Why} does this work? Typically, this will be given as a loop
        invariant proof.  For this HW, explain why it works informally, in your
        own words.

        \paragraph{Answer}
        \todo{}

\end{enumerate}

Note: in all future HWs, if you are asked to come up with an algorithm, you are
expected to give an algorithm that beats the brute force (and, if possible, of
optimal time complexity). With your algorithm, please provide the following:
\begin{itemize}
    \item \emph{What}: A prose explanation of the problem and the algorithm,
        including a description of the input/output.
    \item \emph{How}: Psuedocode, referenced from within the prose explanation.
    \item \emph{How Fast}: Runtime, along with justification.  (Or, in the
        extreme, a proof of termination).
    \item \emph{Why}: Statement of the loop invariant for each loop.
\end{itemize}


%%%%%%%%%%%%%%%%%%%%%%%%%%%%%%%%%%%%%%%%%%%%%%%%%%%%%%%%%%%%%%%%%%%%%%%%%%%%%%
\collab{\todo{}}
\nextprob{Sorting before Searching?}
Let $A$ be an array of $n$ comparable objects.  We do not know if $A$ is sorted
or not.

\begin{enumerate}
    \item To answer the question \emph{is item $x$ in the $A$?}, should we
        sort the array first?  Why or why not?

        \paragraph{Answer}
        \todo{}

    \item Suppose we have $k$ elements that we want to find in $A$. Does this
        change your answer? Why or why~not?

        \paragraph{Answer}
        \todo{}

\end{enumerate}

%%%%%%%%%%%%%%%%%%%%%%%%%%%%%%%%%%%%%%%%%%%%%%%%%%%%%%%%%%%%%%%%%%%%%%%%%%%%%%
\collab{\todo{}}
\nextprob{Recurrence Relations}
Consider the function $T \colon \N \to \N$ defined by
$$T(n) = \begin{cases}
            1        & n=1\\
            T(n-1)+1 & n>1.
         \end{cases}
$$
What is the closed form and asymptotic the form of this recursion?  For the
closed form, use induction to prove that it is correct.  For the asymptotic
form, use the definition of big-Theta to briefly justify (here, you do not need
to submit a second induction, but do submit what constants $c$ and $n_0$ work in the
definition).

\paragraph{Answer}
\todo{}


%%%%%%%%%%%%%%%%%%%%%%%%%%%%%%%%%%%%%%%%%%%%%%%%%%%%%%%%%%%%%%%%%%%%%%%%%%%%%%
\collab{\todo{}}
\nextprob{More Recurrence Relations}

What is the asymptotic form of the following recurrence
relations? (Show work for partial credit, but full justification is not required
on this question).
Let $T \colon \N \to N$ be defined by $T(1)=1$ and, for $n>1$,
\begin{enumerate}
    \item $T(n) = 16 T(n/4) + n$
        \paragraph{Answer} \todo{}
    \item $T(n) = 2 T(n/2) + n \log{n}$
        \paragraph{Answer} \todo{}
    \item $T(n) = 6 T(n/3) + n^2 \log{n}$
        \paragraph{Answer} \todo{}
    \item $T(n) = 4 T(n/2) + n^2$
        \paragraph{Answer} \todo{}
    \item $T(n) = 9 T(n/3) + n$
        \paragraph{Answer} \todo{}
\end{enumerate}

%%%%%%%%%%%%%%%%%%%%%%%%%%%%%%%%%%%%%%%%%%%%%%%%%%%%%%%%%%%%%%%%%%%%%%%%%%%%%%
\collab{\todo{}}
\nextprob{Pancakes}

Chapter 1, Problem 9, parts (a) and (b).

\paragraph{Answer, Part (a)}

\todo{}

\paragraph{Answer, Part (b)}

\todo{}


\end{document}
