\documentclass{article}
\usepackage{../fasy-hw}

%% UPDATE these variables:
\renewcommand{\hwnum}{1}
\title{Advanced Algorithms, Homework \hwnum}
\collab{none}
\author{\todo{Your Name Here}}
\date{due: Wednesday, 11 October 2023}

\begin{document}

\maketitle

This homework assignment should be
submitted as a single PDF file both to D2L and to Gradescope.  The homework is
due at 23:59 on the due date; however, you can continue to submit until the
instructor begins grading, at which point no more submissions will be accepted.

General homework expectations:
\begin{itemize}
    \item Homework should be typeset using LaTex.
    \item Answers should be in complete sentences and proofread.
    \item You will not plagiarize, nor will you share your written solutions
        with classmates.
    \item List collaborators at the start of each question using the
        \texttt{collab} command. If this is still a `TODO' at the time of
        submission, the problem will not
        be graded.
    \item List any outside resources that you used within each problem (not at
        the end of the problem set). If not clear from context, explain how the
        external resource was used.  Outside resources include the use of
        ChatGPT, googling definitions, and looking up answers after you solve
        the problem in order to `check that you got it right'.  {\bf If no outside resources are
        used, the last line of your answer should be ``No outside resources were
        used when working on this problem.''}  Note: in general, the assumption
        is that you do not need any outside resources to answer the problems.
        If you find yourself needing extra resources, it is highly advised to
        ask for them rather than to find them on your own.
    \item Put your answers where the \texttt{todo} command currently is (and
        remove the \texttt{todo}, but not the word \texttt{Answer}).
    \item If you are asked to come up with an algorithm, you are
        expected to give an algorithm that beats the brute force (and, if possible, of
        optimal time complexity). With your algorithm, please provide:
        \begin{itemize}
            \item \emph{What}: A prose explanation of the problem and the algorithm,
                including a description of the input/output.
            \item \emph{How}: Describe how the algorithm works, including giving
                psuedocode for it.  Be sure to reference the pseudocode
                from within the prose explanation.
            \item \emph{How Fast}: Runtime, along with justification.  (Or, at
                the very least, a proof of termination using a decrementing function).
            \item \emph{Why}: Briefly explain why the algorithm works.  Be sure
                to include a statement of the loop invariant for each loop, or
                recursion invariant for each recursive function.
        \end{itemize}
    \item Remove the set of instructions on this page (everything from Line 15
        through Line 59 in the LaTex file.
\end{itemize}

%%%%%%%%%%%%%%%%%%%%%%%%%%%%%%%%%%%%%%%%%%%%%%%%%%%%%%%%%%%%%%%%%%%%%%%%%%%%%%
\collab{\todo{}}
\nextprob{Forgotten Citations}

Please read the instructions on the first page carefully.  This has been
expanded from the last assignment.  (Note the instruction to delete the
instructions).  If you forgot to include proper citations for H-0 or H-1, please
give a detailed list of the missing citations and how they were used in the
homework.

\paragraph{Answer} \todo{}


%%%%%%%%%%%%%%%%%%%%%%%%%%%%%%%%%%%%%%%%%%%%%%%%%%%%%%%%%%%%%%%%%%%%%%%%%%%%%%
\collab{\todo{}}
\nextprob{Short Answers}

Provide short answers to the following questions:

\begin{enumerate}[(a)]
    \item What is the difference between a best-case analysis (which I would
        never ask you to do) and $\Omega$-notation?

        \paragraph{Answer} \todo{}

    \item Suppose you are playing a game with a die.  The game is to roll
        one die. Let $v$ be the value on the top of the die after you roll it.
        You earn or owe (if negative) the following amount: $(-1)^v 2v$.  What
        is the expected amount that you earn in this game?  (If you are expected
        to owe money, report this as a negative number).

        \paragraph{Answer} \todo{}

    \item Let $P(n)$ be some statement that depends on a natural number $n$.  You
        want to prove that $P(n)$ is true for all natural numbers.  To start,
        you prove $P(1)$ is true.  What is the inductive hypothesis?

        \paragraph{Answer} \todo{}

\end{enumerate}


%%%%%%%%%%%%%%%%%%%%%%%%%%%%%%%%%%%%%%%%%%%%%%%%%%%%%%%%%%%%%%%%%%%%%%%%%%%%%%
\collab{\todo{}}
\nextprob{Case 3 of Master's Theorem}

Let $T \colon \N to \N$ be the recursive function defined by $T(n)=a
T(n/b)+f(n)$ with $T(0)=1$.  Case 3 of Master's theorem is as follows:

\emph{IF (1) there exists $\varepsilon \in \R_+$ such that $f(n)
\in O(n^{\log_b a + \varepsilon})$ and (2) there exists $c \in
(0,1)$ and~$n_0 \in \N$ such that for all $n \geq n_0$, $a
f(n/b) \leq c f(n)$, THEN $T(n) \in \Theta(f(n))$.}

Prove that $T(n)=3T(n/4)+ n \log n$ is in Case~3 of Master's Theorem
and give the asymptotic running time.

\paragraph{Answer}
\todo{}


%%%%%%%%%%%%%%%%%%%%%%%%%%%%%%%%%%%%%%%%%%%%%%%%%%%%%%%%%%%%%%%%%%%%%%%%%%%%%%
\collab{\todo{}}
\nextprob{Redo: Pancakes}

Chapter 1, Problem 9, part (a). Please use both the individual feedback and the
classwide feedback in order to improve your response from the submission to H-1.

\paragraph{Answer}
\todo{}


\end{document}
