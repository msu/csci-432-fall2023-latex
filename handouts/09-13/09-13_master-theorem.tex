\documentclass{article}

\usepackage{../handouts}

\title{In-Class Exercise 05}
\author{CSCI 432}
\date{13 September 2023}

\begin{document}
\maketitle

\noindent
Name(s):\\
\\~\\
If not handing in as one group, who did you work with today?

\section*{Master's Theorem}

Master's theorem allows us to quickly solve recurrence relations of the form:
$$ T(n) = a T(n/b) + f(n),$$
where $a, b \in \N$ such that $a \geq 1$ and $b >0$ and $f(n)$ is asymptotically
positive.  Then, we can determine the closed-form of $T(n)$ as follows:
\begin{enumerate}
    \item IF there exists $\varepsilon \in \R_+$ such that $f(n) \in O(n^{\log_b
        a - \varepsilon})$, THEN $T(n) \in \Theta(n^{\log_b a})$.
    \item IF there exists $\varepsilon \in \R_+$ such that $f(n) \in \Theta(n^{\log_b
        a})$, THEN $T(n) \in \Theta(n^{\log_b a}\log n)$.
    \item IF
        \begin{enumerate}[(a)]
            \item there exists $\varepsilon \in \R_+$ such that $f(n) \in \Omega(n^{\log_b
                a + \varepsilon})$ and
            \item there exists $c \in (0,1)$ and $n_0 \in \N$ such that for all $n
                \geq n_0$, the following holds: $a f(n/b) \leq c
                f(n)$,\label{regularity}
        \end{enumerate}
        THEN $T(n) \in \Theta(f(n))$.
\end{enumerate}

\pagebreak

\begin{table}[h!]
    \centering
    \begin{tabular}{|l|l|l|l|l|l|l|l|l|}
        \hline
        &  $a$ & $b$  & $\log_b a$  & $n^{\log_b a}$  & $f(n)$  & Potential
        Case? & $\varepsilon$, if Case 1 or 3  & Closed Form \\ \hline
        \hline
        $T(n) = T(n/2)+1$             & &  &  &  &  & & & \\[5ex] \hline
        $T(n) = 2 T(n/4) + \sqrt{n}$  & &  &  &  & &  & & \\[5ex] \hline
        $T(n) = 2 T(n/4) + n$         & &  &  &  &  & & &  \\[5ex] \hline
        $T(n) = 2 T(n/4) + n^2$       & &  &  &  &  & & & \\[5ex] \hline
        $T(n) = 3 T(n/3) + \Theta(1)$ & &  &  &  & & & & \\[5ex] \hline
    \end{tabular}
\end{table}

Remember, Case 3 has an additional condition to check (this condition
is called the
\emph{regularity condition})! Do that in the space
provided below, or on the back of this page.

\end{document}
