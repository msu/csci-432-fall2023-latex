\documentclass{article}

\usepackage[margin=1in]{geometry}
\usepackage{amsthm,amsmath}

\usepackage{algorithm}
\usepackage{algorithmic}
\renewcommand{\algorithmicrequire}{\textbf{Input:}}
\renewcommand{\algorithmicensure}{\textbf{Output:}}
\renewcommand{\algorithmiccomment}[2]{\hspace{#1}$\triangleright$ {#2} \hfill }

\usepackage{amsfonts}
\def\N{{\mathbb N}}
\def\R{{\mathbb R}}
\def\Z{{\mathbb Z}}

\newcommand{\algref}[1]{Algorithm~\ref{alg:#1}}

\newcommand{\answer}{\textbf{Answer:}\vspace{1.8in}}

\title{Recursion Invariants}
\author{CSCI 432, Fall 2021}

\begin{document}
\maketitle

\section*{Recursion Invariants: Proving Partial Correctness}

Recall the\textsc{Hanoi} algorithm (given in \algref{hanoi});
see~\cite[Ch.~1]{textbook}.

\begin{algorithm}[h!]
    \caption{$\textsc{Hanoi}(n,src,dest,tmp)$}\label{alg:hanoi}
    \begin{algorithmic}[1]
        \REQUIRE $n \in \N$, and three towers with disks: $src,dest,tmp$ such
        that $P$
        \ENSURE $R$ (see below)
        %
        \IF{ $n>0$ }\label{algln:certificate:ifdirs}
            \STATE $\textsc{Hanoi}(n-1,src,tmp,dest)$
            \STATE move top disk from $src$ to $dest$
            \STATE $\textsc{Hanoi}(n-1,tmp,dest,src)$
        \ENDIF
    \end{algorithmic}
\end{algorithm}

\begin{enumerate}
    \item Suppose we have $n$ disks total. What are the assumptions on the input
        to the initial call $\textsc{Hanoi}(n,src,dest,tmp)$?

        \answer

        Going forward, we will call these assumptions $P$.

    \pagebreak
    \item What does it mean for $\textsc{Hanoi}(n,src,dest,tmp)$ to execute
        correctly? What does it return / what does it need to accomplish?

        \answer

        Going forward, we call this statement $Q$.

    \item For a general call to the recursive algorithm what are the assumptions on the input
        (For convenience, suppose the call is: $\textsc{Hanoi}(k,A,B,C)$).

        \answer

        Note: when making a recursive call, we must justify that we have met
        these conditions.

    \item What is the recursion invariant?

        The recursion invariant is statement (i.e., a sentence that can be
        evaluated to TRUE/FALSE).
        In particular, $R$ is (in general) the statement,
        \emph{Each recursive call $\textsc{Hanoi}(k,A,B,C)$ executes
        correctly}.  What does that mean in our case?  Well, it means
        \begin{itemize}
            \item There are currently no violations of smaller disks on larger
                disks. (Note: the world would crumble if this were violated at
                any time), AND
            \item the $k$ smallest disks are now on $B$, AND
            \item no other disks besides the $k$ smallest have moved (since
                right before this call).
        \end{itemize}

        We use the recursion invariant to prove INITIALIZATION, MAINTENANCE, and
        END.  So, sometimes we may see the invariant right away.  Other times,
        we may need to try the proofs then revisit the invariant (as we might
        realize that we forgot something).

    \pagebreak
    \item INITIALIZATION This is like the base case of induction.  Colloquially,
        we ask ``Why is this
        true for the smallest input?'' (And, what are those inputs that
        would allow us to return without a recursive call?)  More formally, we
        can say: If $n_0=0$, then after the call to
        $\textsc{Hanoi}(n_0,A,B,C)$, the recursion invariant $R$ is satisfied.

        \answer

        Note: sometimes, just as in induction, there may be more than one base case.

    \item MAINTENANCE: This part is JUST like the inductive step of induction.
        Let $k \in \N$ such that $k \geq n_0$.  Assume that, for all $k'\in \N$ such that
        $n_0 \leq k' \leq k$, the recursion invariant $R$ holds after a call to
        $\textsc{Hanoi}(k',*,*,*)$. (That was the equivalent to the inductive
        assumption).  Now, we must prove that $R$ holds after
        $\textsc{Hanoi}(k+1,A,B,C)$. (Hint: for this, we will almost always need
        to use the line numbers as we walk
        through the algorithm to explain how the maintenance step works).

        \answer

    \item END: This is where we diverge from induction.  Since algorithms are
        finite, we can't go on forever. Colloquially, we say ``if the initial call
        $\textsc{Hanoi}(n,src,dest,tmp)$ finishes executing, then all $n$ disks
        (which were initially on $src$) have moved to $dst$.'' More formally, we
        can phrase this as: If the recursion invariant holds, if the
        preconditions $P$ are satisfied, and if the algorithm
        completes execution, then the post-condition $Q$ is satisfied.
        (For shorthand, I might write $R \wedge P \wedge T \implies Q$ ).

        \answer

\end{enumerate}

\bibliographystyle{acm}
\bibliography{biblio}

\end{document}
