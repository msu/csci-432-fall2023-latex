\documentclass{article}

\usepackage[margin=1in]{geometry}

\usepackage{algorithm,algorithmic}
\renewcommand{\algorithmicrequire}{\textbf{Input:}}
\renewcommand{\algorithmicensure}{\textbf{Output:}}
\renewcommand{\algorithmiccomment}[2]{\hspace{#1}$\triangleright$ {#2} \hfill }

\usepackage{amsfonts}
\def\R{{\mathbb R}}
\def\N{{\mathbb N}}

\title{Analysis of Algorithms: General Approach}
\author{CSCI 432}

\begin{document}
\maketitle

\noindent
Name:\\
Who did you work with today?

What algorithm are you analyzing today?

\textbf{Answer}
\vspace{1in}

\section{What?}

When analyzing algorithms, we first ask WHAT?  That is, what is the problem that
we want solved?  Typically, in this description, we need to understand both what
are the inputs \emph{and} what are the outputs of the algorithm.  When
describing WHAT, it should be independent of HOW.

So, describe the WHAT for the algorithm you wish to analyze today.

What is the input?

\textbf{Answer}
\vspace{1in}

What is the output?

\textbf{Answer}
\vspace{1in}

Use the previous two answers to concisely describe what is the problem?

\textbf{Answer}
\vspace{1in}

\section{How?}

If you are presenting an algorithm, you should describe how it works in
pseudocode.  More importantly, explain what is going on in the pseudocode in the
prose text.  The prose should point to the algorithm.  (Remember, in LaTex,
algorithms are floating environments, so we need to be reminded to look at them.

\textbf{Answer}
\vspace{3in}

Often, it helps to walk through small examples.  If you came up with the
algorithm, this is a nice sanity check.  If you are studying an algorithm given
to you, this is helpful to better understand the algorithm.

\textbf{Answer}
\vspace{1in}

\pagebreak
\section{How Fast?}

What is the runtime?  If this is a recursive algorithm, you are expected to
explicitly state the recurrence relation.

\textbf{Answer}
\vspace{1in}

Sometimes, it is hard to pin down the runtime, yet we still want to ensure that
the algorithm terminates.  For this, we use decrementing functions to prove that
loops/recursions terminate.
What is the decrementing function for this algorithm?

\textbf{Answer}
\vspace{1in}

Justify (=prove) why function is well-defined.  That is, why is the output of
the function a natural number?

\textbf{Answer}
\vspace{1in}

Justify (=prove) why this decrements each time it reaches the top of the loop
(or each time it goes into a new recursive call).

\textbf{Answer}
\vspace{1in}


\pagebreak
\section{Why? The Loop Invariant}

Finally, we prove why the algorithm works.  Typically, this is done with a loop
(or recursion) invariant.  Here, we focus on setting up the loop invariant.
What
are the following statements (start with your best guess, then come back and
revise if needed):

The loop guard $G$:
\vspace{0.5in}

The post-condition $Q$:
\vspace{0.5in}

The pre-condition $P$:
\vspace{0.5in}

The loop invariant $L=L_i$:
\vspace{0.5in}


Can you use this loop invariant
to prove partial correctness (Remember, there are three parts to partial
correctness: Initialization, Maintenance, and End).

Initialization

\textbf{Answer}
\vspace{1in}


Maintenance

\textbf{Answer}
\vspace{1in}


End

\textbf{Answer}
\vspace{1in}

\end{document}
