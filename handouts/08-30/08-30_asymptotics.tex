\documentclass{article}

\usepackage{../handouts}

\usepackage[letterpaper,margin=1in]{geometry}

\title{CSCI 432 Handout 03: Asymptotic Notation}
\author{Name(s): \rule{3in}{0.15mm} }
\date{30 August 2023}

\begin{document}
\maketitle

\section*{Definitions}

First, let's recall the definitions:

\begin{definition}[Asymptotic Notation]
    Let $f,g \colon \N \to \R$.  Then, we say that \emph{$f$ is $O(g)$} iff: there
    exists $n_0 \in \N$ and $c \in \R_{+}$ such that for all $n \geq n_0$, the
    following holds:
    $$ 0 \leq f(n) \leq cg(n). $$
    This is interpreted as ``$f$ is upper-bounded by $g$.''
    In addition, if those same conditions hold, we also say that~$g$ is $\Omega(f)$.  This is interpreted as
    ``$f$ g is lower-bounded by $f$.''

    The asymptotic tight bound is denoted by $\Theta$.  For $f,g \colon \N \to
    \R$, we say that $f$ is $\Theta(g)$ iff $f$ is $O(g)$ and $f$
    is~$\Omega(g)$. This is the bound that we most often want to find!
\end{definition}

Let's practice!  We will complete some of the following problems
in class.  What we do not
complete, please do them on your own.  These will be collected for attendance,
but not graded. Nonetheless,
but you are expected to know how to answer these questions. If you have any
questions, please reach out to the instructor for help.

\begin{enumerate}
    \item Let $f \colon \N \to \R$ be defined by $f(n) = n^2+2$. Prove that
        $f(n)$ is $O(n^2)$.
        \practice
    \pagebreak
    \item Let $f \colon \N \to \R$ be defined by $f(n) = n^2-2$. Prove that
        $f(n)$ is $O(n^2)$.
        \practice
    \item If $f$ is $\Theta(g)$, is it true or false that $g$ is $\Theta(f)$?
        Why or why not?
        \practice
    \item Prove that $f \colon \N \to \R$ defined by $f(n) = \log_2(n)$ is
        $\Theta(\log_{10}(n)$).
        \practice
    \item Prove that $\Theta$ determines an equivalence relation on functions.
        \practice
    \newpage
    \item We often call $\Theta(1)$ constant and $\Theta(n)$ linear.  We know
        that constant-time algorithms are `faster' than linear-time algorithms.
        But, can you rank the following categories of runtimes?
        $$
        \Theta(n), \Theta(n!), \Theta(1), \Theta(\log n), \Theta(n!),
        \Theta(2^n), \Theta(5^n), \Theta(n^2), \Theta(n^3), \Theta(n \log n)
        $$
        \practice
\end{enumerate}

There are some algorithms whose asymptotic runtimes we should know like the back
of our hands.  Let's see if we can remember (or lookup if we need to) the
runtimes of the following:
{\color{blue}
\begin{enumerate}
    \item Sorting an array\footnote{Yes, this is a problem, not an algorithm.
        When I ask about the runtime of a problem, we think of the
        worst-case runtime of the best solution.}
    \item Searching in a sorted array.
    \item Binary search.
    \item Linear search.
    \item Summing $n$ numbers.
    \item Determining if a number is odd/even.
    \item Finding the $i$-th item in an array.
    \item Finding the $i$-th item in a linked list.
    \item Bubble sort.
    \item Finding all permutations of a set.
    \item Finding all subsets of a set.
\end{enumerate}
}


\end{document}
