\documentclass{article}

\usepackage[margin=1in]{geometry}

\usepackage{algorithm,algorithmic}
\renewcommand{\algorithmicrequire}{\textbf{Input:}}
\renewcommand{\algorithmicensure}{\textbf{Output:}}
\renewcommand{\algorithmiccomment}[2]{\hspace{#1}$\triangleright$ {#2} \hfill }

\usepackage{amsfonts}
\def\R{{\mathbb R}}
\def\N{{\mathbb N}}

\title{Topological Sort: Decrementing Functions and Loop Invariants}
\author{CSCI 432}

\begin{document}
\maketitle

\noindent
Name:\\
Who did you work with today?

\section*{Topological Sort: The Algorithm}

Topological sort takes a poset $(P,\leq)$ and returns a total order of all
objects that is \emph{compatible} with the partial order of the poset.  By
compatible, we mean that for all $a \leq b \in P$, we have $a$ appears before
$b$ in the total order.  For example, consider $P$ to be a set of points in
$\R^2$ and let the partial order be the product order; that is, $(a_x,a_y) \leq
(b_x,b_y)$ if and only if $a_x \leq b_x$ and $a_y \leq b_y$.  Given a set:
$$ P = \{ (0,0), (\pi, \pi), (2,4), (1,2) \}$$
we can return the following total order:
$$[ (0,0), (1,2), (2,4), (\pi, \pi)].$$
Note that this is not the only compatible total order.  What other total order
could have been returned?

\textbf{Answer}
\vspace{1in}

The following algorithm takes as input a poset, represented as a directed
acyclic graph, and returns a compatible total order.  If the poset orders $a
\leq b$, then the DAG has an edge from vertex $a$ to vertex $b$.  For this
algorithm, vertices have at least one attribute: outgoing, which stores a list
of vertices that define the outgoing edges.

\begin{algorithm}[h]
    \caption{Topological Sort}
    %
    \begin{algorithmic}[1]
        \REQUIRE $G=(V,E)$, a DAG

        \ENSURE $L$, a list of vertices in G in a total order compatible with
        the partial order of the DAG.

        \FOR {$v \in V$}
            \STATE Add an attribute `$v$.count' and initialize it to the indegree of $v$.
        \ENDFOR

        \STATE $Q \gets$ the sent of vertices with `count'=0.\label{ln:initQ}

        \STATE $L \gets$ array of length $|V|$
        \STATE $i \gets 1$

        \WHILE {$Q$ is not empty}
            \STATE $v \gets Q.\textsc{POP}$
            \STATE $L[i] \gets v$
            \STATE $i++$
            \FOR {$u \in v$.outgoing}
                \STATE $u$.count$--$
                \IF {$u$.count $=0$}
                    \STATE Add $u$ to $Q$
                \ENDIF
            \ENDFOR
        \ENDWHILE

        \RETURN $L$
    \end{algorithmic}
\end{algorithm}

\pagebreak
In Line~\ref{ln:initQ}, why is $Q$ not empty?

\textbf{Answer}
\vspace{1in}

Explain how topological sort is helpful for dynamic programming.

\textbf{Answer}
\vspace{1in}

Walk through a small example.

\textbf{Answer}

\pagebreak
\section*{Topological Sort: The Decrementing Function}

What is the decrementing function for the for loop?

\textbf{Answer}
\vspace{1in}

What is the decrementing function for the while loop?

\textbf{Answer}
\vspace{1in}

\section*{Topological Sort: The Loop Invariant}

Here, we investigate partial correctness of the while loop.  For this loop, what
are the following statements (start with your best guess, then come back and
revise if needed):

The loop guard $G$:
\vspace{0.5in}

The post-condition $Q$:
\vspace{0.5in}

The pre-condition $P$:
\vspace{0.5in}

The loop invariant $L=L_i$:
\vspace{0.5in}


Can you use this loop invariant
to prove partial correctness (Remember, there are three parts to partial
correctness: Initialization, Maintenance, and End).

\textbf{Answer}
\vspace{1in}

\end{document}
